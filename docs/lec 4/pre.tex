%!TEX TS-program = xelatex

\documentclass[t]{beamer}

\usetheme{Hannover}
\usecolortheme{rose}

\usepackage{fontspec,xltxtra,xunicode}      %% подготавливает загрузку шрифтов Open Type, True Type и др.
%\defaultfontfeatures{Ligatures={TeX},Renderer=Basic}  %% свойства шрифтов по умолчанию
\setmainfont[Ligatures={TeX,Historic},
SmallCapsFont={Brill},
SmallCapsFeatures={Letters=SmallCaps}]{Brill} %% задаёт основной шрифт документа
\setsansfont{Brill}                    %% задаёт шрифт без засечек
\setmonofont[Ligatures=NoCommon]{DejaVu Sans}
\newfontfamily\SYM{Brill}
\usepackage{indentfirst}
%%% Дополнительная работа с математикой
\usepackage{amsmath,amsfonts,amssymb,amsthm,mathtools} % AMS
\usepackage{icomma} % "Умная" запятая: $0,2$ --- число, $0, 2$ --- перечисление

%%% Работа с картинками
\usepackage{wrapfig} % Обтекание рисунков текстом
\usepackage{rotating}
\usepackage{fixltx2e}
\usepackage{hhline}
\usepackage{lscape}

%%% Работа с таблицами
\usepackage{array,tabularx,tabulary,booktabs} % Дополнительная работа с таблицами
\usepackage{longtable}  % Длинные таблицы
\usepackage{multirow} % Слияние строк в таблице

\usepackage{multicol} % Несколько колонок
%%% Страница
%\usepackage{fancyhdr} % Колонтитулы
% 	\pagestyle{fancy}
 	%\renewcommand{\headrulewidth}{0pt}  % Толщина линейки, отчеркивающей верхний колонтитул
% 	\lfoot{Нижний левый}
% 	\rfoot{Нижний правый}
% 	\rhead{Верхний правый}
% 	\chead{Верхний в центре}
% 	\lhead{Верхний левый}
%	\cfoot{Нижний в центре} % По умолчанию здесь номер страницы

\usepackage{setspace} % Интерлиньяж
%\onehalfspacing % Интерлиньяж 1.5
%\doublespacing % Интерлиньяж 2
\singlespacing % Интерлиньяж 1

\usepackage{subfig} % подкартинки
\usepackage{lastpage} % Узнать, сколько всего страниц в документе.
\usepackage{soul} % Модификаторы начертания
\usepackage{bbding}
\usepackage{tikz} % Работа с графикой
\usepackage{pgfplots}
\usepackage{pgfplotstable}
\usepackage{verbatim}

\usepackage{attachfile2}
\usepackage{alltt}

%%% Лингвистические пакеты
%\usepackage{savetrees} % пакет, который экономит место
\usepackage{forest} % для рисования деревьев
\usepackage{vowel} % для рисования трапеций гласных
\usepackage{natbib}
\bibpunct[: ]{[}{]}{;}{a}{}{,}
\usepackage[nogroupskip,nopostdot, nonumberlist]{glossaries}
%\usepackage{glossary-mcols} 
%\setglossarystyle{mcolindex}
\usepackage{philex} % пакет для примеров
\newcommand{\mytem}{\item[$\circ$]}
\addto\captionsrussian{
\renewcommand{\refname}{}}

\newcommand{\apostrophe}{\XeTeXglyph\XeTeXcharglyph"0027\relax}
\usetikzlibrary{patterns}

\usepackage{ulem}

\usepackage{hyperref}